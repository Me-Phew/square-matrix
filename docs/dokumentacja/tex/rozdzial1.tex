\newpage
\section{Ogólne określenie wymagań}		%1
%Określenie celu pracy, co chcemy uzyskać, jakie przewidujemy wyniki

Celem projektu jest stworzenie klasy \texttt{matrix} w języku C++, która pozwala na dynamiczne zarządzanie kwadratowymi macierzami o wymiarach \( n \times n \). Klasa ta powinna zapewniać szeroki zakres funkcjonalności, takich jak alokacja pamięci, operacje na elementach macierzy, generowanie macierzy specjalnych oraz obsługa operatorów. Ostatecznym rezultatem jest stworzenie wydajnego i funkcjonalnego narzędzia do pracy z macierzami, które spełnia wszystkie wymagania zadania.


\subsection{Cel pracy}
Podstawowym celem jest stworzenie modułu \texttt{matrix}, który:
\begin{itemize}
  \item umożliwia dynamiczne alokowanie i zwalnianie pamięci na stercie,
  \item realizuje operacje takie jak dodawanie, mnożenie, transpozycja oraz inne transformacje macierzy,
  \item zapewnia obsługę wyjątków i mechanizmy kontroli poprawności operacji (np. mnożenie macierzy musi spełniać wymagania matematyczne co do wymiarów),
  \item jest w pełni przetestowany poprzez zestaw scenariuszy uruchamianych z poziomu funkcji \texttt{main}.
\end{itemize}

\subsection{Przewidywane wyniki}
Projekt ma dostarczyć:
\begin{itemize}
  \item Wydajną i poprawną implementację klasy \texttt{matrix}, która spełnia wszystkie wymienione wymagania.
  \item Testy sprawdzające poprawność działania klasy dla macierzy o różnych wymiarach, w tym dużych \( n > 30 \).
  \item Dokumentację wygenerowaną przy użyciu Doxygen oraz rozbudowany opis w LaTeX, zawierający analizę funkcjonalności i trudności napotkanych podczas implementacji.
  \item Praktyczne doświadczenie w wykorzystaniu GitHub Copilot jako narzędzia wspierającego programowanie oraz w pracy zespołowej z użyciem systemu kontroli wersji Git.
\end{itemize}

\subsection{Zakres projektu}
Projekt obejmuje:
\begin{itemize}
  \item Implementację klasy \texttt{matrix} w języku C++ w osobnym pliku źródłowym.
  \item Funkcję \texttt{main} testującą wszystkie zaimplementowane funkcjonalności.
  \item Mechanizmy alokacji dynamicznej, zarządzania pamięcią i zabezpieczenia przed błędnymi operacjami.
  \item Obsługę przeciążonych operatorów matematycznych dla macierzy.
  \item Dokumentację w formacie LaTeX oraz automatycznie generowaną dokumentację kodu przy pomocy Doxygen.
  \item Analizę wykorzystania narzędzi AI w procesie tworzenia projektu oraz wpływ AI na efektywność pracy.
\end{itemize}

Projekt ten pozwoli nie tylko na zdobycie praktycznych umiejętności programistycznych, ale również na pogłębienie wiedzy w zakresie tworzenia dokumentacji oraz efektywnej pracy zespołowej.
