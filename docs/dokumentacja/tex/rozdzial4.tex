\newpage
\section{Implementacja}		%4
%Opisać implementacje algorytmu/programu. Pokazać ciekawe fragmenty kodu
%Opisać powstałe wyniki (algorytmu/nrzędzia)
W tej sekcji przedstawiono szczegóły implementacji klasy \texttt{matrix}, omówiono najważniejsze fragmenty kodu oraz wyniki uzyskane w trakcie działania programu.

\subsection{Opis implementacji}
Klasa \texttt{matrix} została zaprojektowana z myślą o dynamicznym zarządzaniu pamięcią dla kwadratowych macierzy o wymiarach \(n \times n\). Kluczowe elementy implementacji obejmują:
\begin{itemize}
  \item \textbf{Konstruktor i destruktor:}
        Klasa wykorzystuje konstruktor do alokacji pamięci oraz destruktor do jej zwalniania, co zapobiega wyciekom pamięci.
  \item \textbf{Operacje matematyczne:}
        Klasa obsługuje dodawanie, mnożenie i transpozycję macierzy. Metody te są zaimplementowane w sposób uwzględniający zgodność wymiarów macierzy.
  \item \textbf{Walidacja danych:}
        Operacje na macierzach są zabezpieczone przed nieprawidłowymi operacjami, takimi jak próba mnożenia macierzy o różnych wymiarach.
\end{itemize}

\subsection{Fragmenty kodu}
Poniżej zaprezentowano najważniejsze fragmenty kodu, które ilustrują implementację wybranych funkcji:

\subsubsection{Konstruktor klasy \texttt{matrix}}
Konstruktor klasy odpowiada za dynamiczną alokację pamięci \textbf{(Listing 1)}:
\begin{lstlisting}[language=C++, caption={Konstuktor klasy}, label={lst:operator}]
Matrix::Matrix(size_t n) : size(n) {
    data = new double*[size];
    for (size_t i = 0; i < size; ++i) {
        data[i] = new double[size]();
    }
}
\end{lstlisting}

\subsubsection{Operacja mnożenia macierzy}
Poniżej przedstawiono implementację funkcji realizującej mnożenie dwóch macierzy \textbf{(Listing 2)}:
\begin{lstlisting}[language=C++, caption={Implementacja funkcji \texttt{operator}}, label={lst:operator}]
Matrix Matrix::operator*(const Matrix& other) const {
    if (size != other.size) {
        throw std::invalid_argument("Incompatible matrix dimensions.");
    }
    Matrix result(size);
    for (size_t i = 0; i < size; ++i) {
        for (size_t j = 0; j < size; ++j) {
            for (size_t k = 0; k < size; ++k) {
                result.data[i][j] += data[i][k] * other.data[k][j];
            }
        }
    }
    return result;
}
\end{lstlisting}

\subsubsection{Transpozycja macierzy}
Funkcja transponuje macierz, zamieniając wiersze z kolumnami \textbf{(Listing 3)}:
\begin{lstlisting}[language=C++, caption={Implementacja fukncji \texttt{Transpose}}, label={lst:transpose}]
Matrix Matrix::transpose() const {
    Matrix result(size);
    for (size_t i = 0; i < size; ++i) {
        for (size_t j = 0; j < size; ++j) {
            result.data[j][i] = data[i][j];
        }
    }
    return result;
}
\end{lstlisting}

\newpage

\subsection{GitHub Copilot}

\subsection{Trudności i ograniczenia w pracy z AI}

W trakcie implementacji z użyciem Copilot napotkaliśmy zarówno zalety, jak i wyzwania związane z wykorzystaniem sztucznej inteligencji. Poniżej podsumowano kluczowe aspekty współpracy z Copilotem.

\subsubsection{W czym AI sobie nie radziła?}

Podczas pracy z Copilotem zauważyliśmy kilka problemów:

\begin{enumerate}
  \item \textbf{Nieprecyzyjne sugestie w bardziej złożonych algorytmach.} \\
        AI często proponowała błędne implementacje bardziej skomplikowanych funkcji, takich jak obliczanie wyznacznika macierzy czy mnożenie macierzy. Na przykład pierwotnie wygenerowany kod dla mnożenia macierzy nie uwzględniał sumowania elementów w odpowiednich iteracjach \textbf{(Listing 4)}:

        \begin{lstlisting}[language=C++, caption=Błędna implementacja mnożenia macierzy]
    SquareMatrix multiply(const SquareMatrix &other) const {
        if (size != other.size) {
            throw std::invalid_argument("Matrix dimensions must match");
        }
        SquareMatrix result(size);
        for (int i = 0; i < size; ++i) {
            for (int j = 0; j < size; ++j) {
                result.matrix[i][j] = matrix[i][j] * other.matrix[i][j]; // BLAD
            }
        }
        return result;
    }
    \end{lstlisting}

        W tym przypadku AI błędnie uznała, że mnożenie elementów w tej samej pozycji w wierszach i kolumnach wystarczy, zamiast zastosować klasyczny algorytm mnożenia macierzy.

        Poprawna implementacja wymagała ręcznej ingerencji \textbf{(Listing 5)}:

        \begin{lstlisting}[language=C++, caption=Poprawna implementacja mnożenia macierzy]
    SquareMatrix multiply(const SquareMatrix &other) const {
        if (size != other.size) {
            throw std::invalid_argument("Matrix dimensions must match");
        }
        SquareMatrix result(size);
        for (int i = 0; i < size; ++i) {
            for (int j = 0; j < size; ++j) {
                result.matrix[i][j] = 0;
                for (int k = 0; k < size; ++k) {
                    result.matrix[i][j] += matrix[i][k] * other.matrix[k][j];
                }
            }
        }
        return result;
    }
    \end{lstlisting}

  \item \textbf{Nieoptymalne zarządzanie pamięcią.} \\
        Copilot czasami sugerował wykorzystanie statycznych tablic zamiast dynamicznej alokacji pamięci, co ograniczało skalowalność programu. Ręczna refaktoryzacja była konieczna, aby zastosować elastyczną strukturę, taką jak \texttt{std::vector}.
\end{enumerate}

\subsubsection{Jakie błędy popełniała AI?}

\begin{itemize}
  \item \textbf{Nieprawidłowa obsługa wyjątków:} W wygenerowanym kodzie brakowało odpowiednich rzutów wyjątków w funkcjach, co prowadziło do nieprzewidzianego zachowania w sytuacjach brzegowych.
  \item \textbf{Niejasne komentarze:} AI generowała komentarze, które były zbyt ogólne i mało pomocne, np. \texttt{// Add matrices}, co wymagało ręcznej edycji.
\end{itemize}

\subsubsection{W czym AI pomogła?}

Mimo trudności, Copilot znacząco przyspieszył proces tworzenia kodu poprzez:
\begin{itemize}
  \item \textbf{Automatyzację prostych funkcji:} AI dobrze radziła sobie z generowaniem podstawowych metod, takich jak akcesory (\texttt{getElement} i \texttt{setElement}) oraz funkcji wyświetlających macierz.
  \item \textbf{Sugestie składniowe:} W wielu przypadkach Copilot znacząco ułatwił tworzenie pętli i warunków dzięki precyzyjnym sugestiom składniowym.
\end{itemize}

Przykład wygenerowanej poprawnej funkcji wyświetlania macierzy \textbf{(Listing 6)}:
\begin{lstlisting}[language=C++, caption=Funkcja wyświetlająca macierz]
void display() const {
    for (const auto &row : matrix) {
        for (const auto &element : row) {
            std::cout << element << " ";
        }
        std::cout << std::endl;
    }
}
\end{lstlisting}

\subsection{Wnioski z pracy z Copilotem}

Praca z AI, takim jak Copilot, była cennym doświadczeniem, które pokazało zarówno mocne, jak i słabe strony tego narzędzia. Sztuczna inteligencja okazała się bardzo przydatna przy generowaniu kodu dla prostych operacji oraz w sytuacjach, gdy kluczowa była znajomość składni języka C++. Jednak w przypadku bardziej złożonych algorytmów konieczne było ręczne wprowadzanie poprawek, a także samodzielne projektowanie struktury programu.

W kolejnych projektach warto korzystać z Copilota jako narzędzia wspomagającego, ale z zachowaniem krytycznego podejścia do wygenerowanego kodu.

\subsection{Wyniki działania programu}
Przeprowadzone testy jednostkowe w środowisku Google Test wykazały, że:
\begin{itemize}
  \item Operacje matematyczne, takie jak dodawanie, mnożenie i transpozycja, zostały zaimplementowane poprawnie.
  \item Program prawidłowo obsługuje sytuacje wyjątkowe, np. próby mnożenia macierzy o różnych wymiarach.
  \item Algorytm efektywnie działa dla macierzy o wymiarach do \(1000 \times 1000\).
\end{itemize}

\subsection{Podsumowanie}
Implementacja klasy \texttt{matrix} została zrealizowana zgodnie z założeniami projektowymi. Dzięki zastosowaniu dynamicznej alokacji pamięci oraz obsłudze wyjątków, program jest elastyczny i odporny na nieprzewidziane błędy. Wyniki testów potwierdzają poprawność oraz wydajność zaimplementowanego algorytmu.
