\newpage
\section{Projektowanie}		%3
%Napisać z jakich narzędzi będziemy korzystać (kompilator, język programowania), git, biblioteki dodatkowe, itp.
%Opisać szczegółowe ustawienia kompilatora (jeśli są), powiązania z bibliotekami, itp.
%Narysować graf, UML, diagram klas, schemat działania algorytmu
%Jeśli zadanie zakłada przedstawienie jakiegoś narzędzia (np. git, AI) należy opisać sposób jego używania
W projekcie implementacji klasy \texttt{matrix} wykorzystano następujące narzędzia:

\subsection{Narzędzia programistyczne}
\begin{itemize}
  \item \textbf{Język programowania i kompilator:}
        Projekt został napisany w języku \textbf{C++}, a do kompilacji kodu wykorzystano narzędzie \textbf{CMake}, które umożliwia zarządzanie konfiguracją procesu budowania oraz integrację z bibliotekami zewnętrznymi.
  \item \textbf{Google Test:}
        W celu przetestowania poprawności implementacji metod klasy \texttt{matrix} użyto frameworka \textbf{Google Test}. Testy obejmują weryfikację operacji matematycznych (np. dodawanie, mnożenie) oraz transformacji macierzy (np. transpozycja).
  \item \textbf{Doxygen:}
        Do generowania dokumentacji kodu źródłowego wykorzystano narzędzie \textbf{Doxygen}, które pozwala na automatyczne tworzenie szczegółowego opisu funkcji, zmiennych i struktur projektu.
\end{itemize}

\subsection{Kontrola wersji}
Kod źródłowy projektu jest zarządzany za pomocą systemu kontroli wersji \textbf{Git}, a repozytorium znajduje się na platformie \textbf{GitHub} pod adresem:
\url{https://github.com/Me-Phew/square-matrix}.
Zastosowano standardowe praktyki zarządzania kodem, takie jak regularne commitowanie zmian i utrzymywanie głównej gałęzi w stanie gotowym do użycia.

\subsection{Specyfikacja kompilatora}
W projekcie wykorzystano następujące ustawienia kompilatora:
\begin{itemize}
  \item Standard języka: \textbf{C++17}.
  \item Plik konfiguracyjny \texttt{CMakeLists.txt} definiuje właściwości kompilacji oraz linkowanie z biblioteką Google Test.
\end{itemize}

\subsection{Sposób użycia narzędzi AI}
W trakcie implementacji projektu wspierano się narzędziem \textbf{GitHub Copilot}, które oferowało sugestie dotyczące składni i implementacji funkcji. Narzędzie to było szczególnie pomocne w przyspieszeniu procesu tworzenia kodu oraz w poprawie czytelności. Szczegóły wykorzystania AI oraz analiza jego efektywności zostaną omówione w kolejnym rozdziale.

